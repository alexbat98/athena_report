\usepackage{fontspec}
\usepackage{xunicode}
\usepackage{xltxtra}
\usepackage{pdfpages}
\usepackage{titling}
\usepackage{xltxtra}
\usepackage{indentfirst} % Красная строка после заголовка
\usepackage{enumerate}
\usepackage{amsmath}
\usepackage[perpage]{footmisc}

\defaultfontfeatures{Ligatures=TeX}
\setmainfont{Times New Roman}
\newfontfamily\cyrillicfont{Times New Roman}

\renewcommand{\baselinestretch}{1.4} % Полуторный межстрочный интервал
\parindent 1.25cm % Абзацный отступ

\usepackage{titlesec}

\titleformat{\section}
  {\normalfont\fontsize{16}{18}\bfseries}
  {\thesection.}
  {1em}
  {}
\titleformat{\subsection}
  {\normalfont\fontsize{14}{16}\bfseries}
  {\thesubsection.}
  {1em}
  {}
\titleformat{\subsubsection}
  {\normalfont\fontsize{12}{14}\bfseries}
  {\thesubsubsection.}
  {1em}
  {}

\titlespacing\section{0pt}{0pt}{12pt}
\titlespacing\subsection{0pt}{24pt}{12pt}
\titlespacing\subsubsection{0pt}{24pt}{12pt}

% Оформление библиографии и подрисуночных записей через точку
\makeatletter
\renewcommand*{\@biblabel}[1]{\hfill#1.}
\renewcommand*\l@section{\@dottedtocline{1}{1em}{1em}}
\renewcommand{\thefigure}{\thesection.\arabic{figure}} % Формат рисунка секция.номер
\renewcommand{\thetable}{\thesection.\arabic{table}} % Формат таблицы секция.номер
\def\redeflsection{\def\l@section{\@dottedtocline{1}{0em}{10em}}}
\makeatother

\usepackage{tocloft}
\renewcommand{\cftsecfont}{\hspace{0pt}}            % Имена секций в содержании не жирным шрифтом
\renewcommand\cftsecleader{\cftdotfill{\cftdotsep}} % Точки для секций в содержании
\renewcommand\cftsecpagefont{\mdseries}             % Номера страниц не жирные
\setcounter{tocdepth}{3}                            % Глубина оглавления, до subsubsection

\sloppy             % Избавляемся от переполнений
\hyphenpenalty=1000 % Частота переносов
\clubpenalty=10000  % Запрещаем разрыв страницы после первой строки абзаца
\widowpenalty=10000 % Запрещаем разрыв страницы после последней строки абзаца

% Списки
\usepackage{enumitem}
\setlist[enumerate,itemize]{leftmargin=12.7mm} % Отступы в списках

\makeatletter
    \AddEnumerateCounter{\asbuk}{\@asbuk}{м)}
\makeatother
\setlist{nolistsep} % Нет отступов между пунктами списка
\renewcommand{\labelitemi}{--} % Маркет списка --
\renewcommand{\labelenumi}{\asbuk{enumi})} % Список второго уровня
\renewcommand{\labelenumii}{\arabic{enumii})} % Список третьего уровня
