\usepackage{xcolor}
\usepackage{listings} % Оформление исходного кода
%\lstset{
%    basicstyle=\small\ttfamily, % Размер и тип шрифта
%    breaklines=true, % Перенос строк
%    tabsize=2, % Размер табуляции
%    literate={--}{{-{}-}}2 % Корректно отображать двойной дефис
%}

\definecolor{stringColor}{rgb}{0.75, 0.38, 0.42}
\definecolor{keywordsColor}{rgb}{0.37, 0.51, 0.67}
\definecolor{commentsColor}{rgb}{0.50, 0.63, 0.76}

\newfontfamily{\jbmono}{JetBrains Mono}

\lstset{ %
  backgroundcolor=\color{white},   % choose the background color; you must add \usepackage{color} or \usepackage{xcolor}
  basicstyle=\footnotesize\jbmono,        % the size of the fonts that are used for the code
  breakatwhitespace=false,         % sets if automatic breaks should only happen at whitespace
  breaklines=true,                 % sets automatic line breaking
  captionpos=t,                    % sets the caption-position to bottom
  commentstyle=\color{commentsColor}\textit,    % comment style
  deletekeywords={...},            % if you want to delete keywords from the given language
  escapeinside={\%*}{*)},          % if you want to add LaTeX within your code
  extendedchars=true,              % lets you use non-ASCII characters; for 8-bits encodings only, does not work with UTF-8
  frame=tb,	                   	   % adds a frame around the code
  keepspaces=true,                 % keeps spaces in text, useful for keeping indentation of code (possibly needs columns=flexible)
  keywordstyle=\color{keywordsColor}\bfseries,       % keyword style
  language=Python,                 % the language of the code (can be overrided per snippet)
  otherkeywords={*,...},           % if you want to add more keywords to the set
  numbers=left,                    % where to put the line-numbers; possible values are (none, left, right)
  numbersep=5pt,                   % how far the line-numbers are from the code
  numberstyle=\tiny\color{commentsColor}, % the style that is used for the line-numbers
  rulecolor=\color{black},         % if not set, the frame-color may be changed on line-breaks within not-black text (e.g. comments (green here))
  showspaces=false,                % show spaces everywhere adding particular underscores; it overrides 'showstringspaces'
  showstringspaces=false,          % underline spaces within strings only
  showtabs=false,                  % show tabs within strings adding particular underscores
  stepnumber=1,                    % the step between two line-numbers. If it's 1, each line will be numbered
  stringstyle=\color{stringColor}, % string literal style
  tabsize=2,	                   % sets default tabsize to 2 spaces
  title=\lstname,                  % show the filename of files included with \lstinputlisting; also try caption instead of title
  columns=fixed                    % Using fixed column width (for e.g. nice alignment)
}
\lstdefinelanguage{llvm}{
	sensitive=true,
	alsoletter={\%},
	% comments.
	%    ; line comment
	comment=[l]{;},
	% strings.
	%    "foo"
	string=[b]{"},
	% instructions.
	%    ref: http://llvm.org/docs/LangRef.html#instruction-reference
	keywords=[1]{
add, addrspacecast, alloca, and, ashr, atomicrmw, bitcast, br, call, cmpxchg,
extractelement, extractvalue, fadd, fcmp, fdiv, fence, fmul, fpext, fptosi,
fptoui, fptrunc, frem, fsub, getelementptr, icmp, indirectbr, insertelement,
insertvalue, inttoptr, invoke, landingpad, load, lshr, mul, or, phi, ptrtoint,
resume, ret, sdiv, select, sext, shl, shufflevector, sitofp, srem, store, sub,
switch, to, trunc, udiv, uitofp, unreachable, urem, va_arg, xor, zext
	},
	% directives.
	%    ref: http://llvm.org/docs/LangRef.html
	keywords=[2]{
acq_rel, acquire, addrspace, alias, align, alignstack, alwaysinline, any,
anyregcc, appending, arcp, arm_aapcs_vfpcc, arm_aapcscc, arm_apcscc, asm,
atomic, attributes, available_externally, blockaddress, builtin, byval, c,
catch, cc, ccc, cleanup, cold, coldcc, comdat, common, constant, datalayout,
declare, default, define, dereferenceable, dllexport, dllimport, eq, exact,
exactmatch, extern_weak, external, externally_initialized, false, fast, fastcc,
filter, gc, ghccc, global, hidden, inalloca, inbounds, initialexec, inlinehint,
inreg, intel_ocl_bicc, inteldialect, internal, jumptable, largest, linkonce,
linkonce_odr, localdynamic, localexec, max, min, minsize, module, monotonic,
msp430_intrcc, musttail, naked, nand, ne, nest, ninf, nnan, noalias, nobuiltin,
nocapture, noduplicate, noduplicates, noimplicitfloat, noinline, nonlazybind,
nonnull, noredzone, noreturn, nounwind, nsw, nsz, null, nuw, oeq, oge, ogt, ole,
olt, one, opaque, optnone, optsize, ord, personality, prefix, preserve_allcc,
preserve_mostcc, private, prologue, protected, ptx_device, ptx_kernel, readnone,
readonly, release, returned, returns_twice, samesize, sanitize_address,
sanitize_memory, sanitize_thread, section, seq_cst, sge, sgt, sideeffect,
signext, singlethread, sle, slt, spir_func, spir_kernel, sret, ssp, sspreq,
sspstrong, tail, target, thread_local, triple, true, type, ueq, uge, ugt, ule,
ult, umax, umin, undef, une, unnamed_addr, uno, unordered, unwind, uselistorder,
uselistorder_bb, uwtable, volatile, weak, weak_odr, webkit_jscc, x,
x86_64_sysvcc, x86_64_win64cc, x86_fastcallcc, x86_stdcallcc, x86_thiscallcc,
x86_vectorcallcc, xchg, zeroext, zeroinitializer
	},
	% types.
	%    ref: http://llvm.org/docs/LangRef.html#type-system
	keywords=[3]{
i1, i2, i3, i4, i5, i6, i7, i8, i9, i10, i11, i12, i13, i14, i15, i16, i17, i18,
i19, i20, i21, i22, i23, i24, i25, i26, i27, i28, i29, i30, i31, i32, i33, i34,
i35, i36, i37, i38, i39, i40, i41, i42, i43, i44, i45, i46, i47, i48, i49, i50,
i51, i52, i53, i54, i55, i56, i57, i58, i59, i60, i61, i62, i63, i64, i80, i512,
void, half, float, double, fp128, x86_fp80, ppc_fp128, x86_mmx, label, metadata
	},
}
\lstdefinelanguage{athgraph}{
	sensitive=true,
	alsoletter={\%, .},
	% morestring=[b]",
	morecomment=[l]{//},
	keywords=[1]{
    module, ath_graph.node, ath_graph.graph, ath_graph.create_tensor,
    ath_graph.alloc, ath_graph.lock, ath_graph.invoke_loader,
    ath_graph.release, ath_graph.return, ath_graph.add, constant,
    ath_graph.eval, ath_graph.barrier, ath_graph.graph_terminator,
		return, ath_rt.alloc, ath_rt.select_device, ath_rt.release,
		ath_rt.null_event, ath_rt.launch
	},
	% directives.
	%    ref: http://llvm.org/docs/LangRef.html
	keywords=[2]{
		virtual_address, node_id, cluster_id, lock_type, sym_name, clusterId
	},
	% types.
	%    ref: http://llvm.org/docs/LangRef.html#type-system
	keywords=[3]{
i1, i2, i3, i4, i5, i6, i7, i8, i9, i10, i11, i12, i13, i14, i15, i16, i17, i18,
i19, i20, i21, i22, i23, i24, i25, i26, i27, i28, i29, i30, i31, i32, i33, i34,
i35, i36, i37, i38, i39, i40, i41, i42, i43, i44, i45, i46, i47, i48, i49, i50,
i51, i52, i53, i54, i55, i56, i57, i58, i59, i60, i61, i62, i63, i64, i80, i512,
void, half, float, double, fp128, x86_fp80, ppc_fp128, x86_mmx, label, metadata,
tensor
	},
}

\lstloadlanguages{llvm, athgraph}
\renewcommand{\lstlistingname}{Листинг}
