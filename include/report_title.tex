{\setstretch{1.0}
\begin{titlepage}
  \begin{center}
    МИНИСТЕРСТВО ОБРАЗОВАНИЯ И НАУКИ РОССИЙСКОЙ ФЕДЕРАЦИИ\break
    Федеральное государственное автономное образовательное учреждение высшего образования\break
    \textbf{«Национальный исследовательский Нижегородский государственный университет им.~Н.И.~Лобачевского» (ННГУ)}
    \break

    \vspace*{1.25cm}

    \textbf{Институт информационных технологий, математики и механики}\break
    \textbf{Кафедра: Математического обеспечения суперкомпьютерных технологий}
    \vspace{0.5cm}

    Направление подготовки: «Фундаментальная информатика и информационные технологии»\break

    \vspace{2.5cm}

    \large{\textbf{ОТЧЕТ}}\break
    по учебной практике\break

    \vspace{0.25cm}

    на тему:\break
    \large{\textbf{«Фреймворк для задач искуственного интеллекта. Вируальная машина.»}}
  \end{center}

\vspace{2cm}

\hfill\textbf{Выполнил:} студент группы 381606-3

 \hfill Баташев Александр Юрьевич

 \hfill\textbf{Научный руководитель:} профессор, д.т.н.

 \hfill Турлапов Вадим Евгеньевич
\vfill
\begin{center}
  Нижний Новгород\break
  2018
\end{center}
\end{titlepage}
}
