\clearpage
\section{Заключение}

Компиляторы графов вычислений -- относительно новое явление в искусственном
интеллекте. Крупные производители пытаются <<выжать>> максимум из оборудования,
которое есть сегодня на рынке. Такие компании как Facebook и Google развивают
собственные разработки Glow и MLIR. Это позволяет им эффективно обрабатывать
постоянно увеличивающиеся потоки данных.

Другое важное направление в развитии компиляторов для искусственного обучения --
выполнение вычислений на устройствах пользователей. На конференции Google I/O
2019 компания Google рассказала о технологии Federated learning
\footnote{https://www.youtube.com/watch?v=89BGjQYA0uE}, которую она развивает
уже несколько лет. Для корпораций распределенное машинное обучение означает
снижение расходов на обслуживание дата-центров. Для пользователей -- повышение
приватности и сохранности данных. Поскольку в качестве конечных точек
все чаще выступают мобильные устройства, особое внимание уделяется
энергопотреблению.
