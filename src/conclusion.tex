\clearpage
\section{Заключение}

В ходе работы были исследованы фундаментальные концепции лежащие в основе
современных фреймворков искусственного интеллекта, их архитектурные особенности.
Современные библиотеки машинного обучения -- сложные многоуровневые системы.
Они предоставляют различные слои абстракции, необходимые для построения и
развертывания систем на основе глубокого обучения. Предоставляя с одной стороны
простой и понятный интерфейс для исследователя, фреймворки скрывают от пользователя
множество особенностей целевой аппаратной платформе, что позволяет, сосредоточившись
на решении конкретной проблемы, получить достойную производительность на разных
компьютерных системах.

Кроме того, была изучена возможность применения технологий разработки и
построения компиляторов в машинном обучении. Компиляторы -- мост между
программным и аппаратным обеспечением. Многолетний опыт, накопленный в ходе
разработки классических оптимизирующих компиляторов, успешно применяется при
построении фреймворков искусственного интеллекта. Это позволяет одновременно
увеличить производительность на уже поддерживаемых платформах и перенести код
на новые платформы с минимальными изменениями. Более того, машинное обучение
заставило исследователей обобщить используемые методы оптимизации, сделав
их более точными и эффективными. Это, в свою очередь, положительно сказалось
на общей производительности кода, генерируемого компиляторами.

Изученные принципы и технологии легли в основу компилятора графа вычислений, что
стало важным практическим достижением работы. Использование инфраструктуры MLIR 
для построения диалектов промежуточного представления позволяет с одной стророны
сохранять на каждом этапе достаточное количество семантической информации для
проведения необходимых оптимизаций, а сдругой стороны двигаться поступательно,
избавляясь от абстракций, когда в них больше нет нужды. Фреймворк LLVM
зарекомендовал себя как удобное и надежное решение для исследователей в сфере
компиляторных технологий. С его помощью легко построить прототип, который в
дальшейшем можно развить до готового коммерческого решения. Причина этому --
модульная архитектура и большое разнообразие поддерживаемых платформ, а так же
активное сообщество разработчиков, к которому присоединились многие крупные
производители программного и аппаратного обеспечения.

Итоговый программный продукт демонстрирует
новый подход, который только начали брать на вооружение коммерческие компании. 
Он может стать базой для построения эффективных инструментов для задач 
искусственного интеллекта. Будучи центральным компонентом современных 
фреймворков, граф вычислений и его реализация оказывают большое влияние на 
производительность нейронных сетей. Кроме того, графы вычислений могут 
встречаться и в других сферах, например, при глобальной оптимизации функций или
решении дифференциальных уравнений.

В дальнейшем планируется развивать получившийся компилятор графа вычислений.
Среди перспективных направлений более <<агрессивные>> оптимизации, например,
улучшения в области слияния вершин и операций, генерация кода операций с учетом
особенностей аппаратного обеспечения, которое используется для исполнения,
усовершенствованные механизмы планирования исполнения вершин.

Также необходимо отметить, что первоначальная гипотеза об актуальности и важности
синтеза компиляторов и средств машинного обучения подтвердилась. За время
проведения исследования свои наработки и достижения в этой области представили
такие компании как Google (доклад <<MLIR Primer: A Compiler Infrastructure for 
the End of Moore’s Law>> на конференции CGO 2019\footnote{Слайды доступны по 
адресу https://bit.ly/2yDgpG0}) и Intel (доклад <<nGraph Dialect: High-Level 
Graph Optimizations for Deep Learning Workloads>>\footnote{Слайды доступны по 
адресу https://bit.ly/2L7VDks}).

