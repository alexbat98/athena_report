\clearpage
\begingroup
\phantomsection
\addcontentsline{toc}{section}{Приложение 1. Тексты программных модулей}
\section*{Приложение 1. Тексты программных модулей}

\phantomsection
\addcontentsline{toc}{subsection}{П1.1  Результат работы различных стадий компилятора Clang}
\subsection*{П1.1. Результат работы различных стадий компилятора Clang}
\lstinputlisting[label={lst:clang_ast},caption={Текстовое представление Clang AST}]{listings/factorial.ast}

\lstinputlisting[language=llvm,label={lst:llvm_ir_clang},caption={LLVM IR, сгенерированный Clang}]{listings/factorial.ll}

\phantomsection
\addcontentsline{toc}{subsection}{П1.2  Результат работы различных стадий компилятора графа вычислений}
\subsection*{П1.2. Результат работы различных стадий компилятора графа вычислений}
\lstinputlisting[language=athgraph,label={lst:ath_graph_add},caption={Код на языке высокоуровневого диалекта для функции сложения}]{listings/add_graph.mlir}
\lstinputlisting[language=athgraph,label={lst:ath_rt_add},caption={Код на языке низкоуровневого диалекта для функции сложения}]{listings/add_rt.mlir}
\lstinputlisting[language=llvm,label={lst:llvm_add},caption={Итоговый LLVM IR для функции сложения}]{listings/add_llvm.ll}
\lstinputlisting[language=athgraph,label={lst:logreg_graph},caption={Код на языке высокоуровневого диалекта для модели логистической регрессии}]{listings/logreg_graph.mlir}
