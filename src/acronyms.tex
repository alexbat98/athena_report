\clearpage

\begingroup
\phantomsection
\addcontentsline{toc}{section}{Список сокращений и условных обозначений}  % Добавляем его в оглавление
\section*{Список сокращений и условных обозначений}
\noindent

\addtocounter{table}{-1}% Нужно откатить на единицу счетчик номеров таблиц, так как следующая таблица сделана для удобства представления информации по ГОСТ
\begin{longtabu} to \textwidth {r X}
\textbf{ОЗУ} & Оперативное запоминающее устройство \\
\textbf{СНС} & Сверточная нейронная сеть \\
\textbf{AST} & Abstract syntax tree, абстрактное синтаксическое дерево \\
\textbf{CPU} & Central Processing Unit, центральное процессорное устройство \\
\textbf{FPGA} & Field-programmable Gate Array, программируемая пользователем
вентильная матрица \\
\textbf{GPU} & Graphics Processing Unit, графическое процессорное устройство \\
\textbf{IR} & Intermediate Representation, промежуточное представление \\
\textbf{JIT} & Just-in-time компиляция, компиляция точно в срок -- технология
компиляции исходного (или промежуточного) кода непосредственно перед исполнением \\
\textbf{LLVM} & Ранее рашифровывалось как Low Level Virtual Machine, фреймворк 
для построения компиляторов, включающий в себя набор стандартных оптимизаций и 
инструменты генерации машинных иструкций для разных платформ \\
\textbf{LSTM} & Long Short-Term Memory, доглая краткосрочная память,
разновидность архитектуры рекуррентных нейронных сетей \\
\textbf{MLIR} & Multi-level Intermediate Representation, фреймворк для построения
промежуточных представлений и оптимизаций над ними \\
\textbf{SSA} & Single static assignment, промежуточное представление,
используемое компиляторами, в котором каждой переменной значение присваивается
лишь единожды \\
\end{longtabu}
